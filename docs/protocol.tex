\documentclass[11pt]{article}
\usepackage[utf8]{inputenc}
\usepackage{polski}
\usepackage{english}
\title{Tamandua Protocol}
\author{Kacper Kołodziej\\ kacper@kolodziej.in}
\date{23th October 2014}
\begin{document}
\maketitle
\makeindex

\section{Header}

Header of Tamandua's message contain these fields:
\begin{itemize}
	\item id -- 4 bytes
	\item type -- 1 byte
	\item author -- 4 bytes
	\item group -- 2 bytes
	\item error code -- 1 byte
	\item UTC time -- 4 bytes
	\item size -- 4 bytes
	\item options -- 2 bytes
	\item author's identification -- 64 bytes
	\item client's identification -- 128 bytes
\end{itemize}

\subsection{Id}

Id field contains next number assigned by server's component called MessageManager. Each message has its own, unique number which allows us to identify message. Size of 4 bytes allows to mark $2^{32} \approx 4.3$ billions of messages. By default its equal zero. Server assigns value to this field.

\subsection{Type}

Type field contains 1 byte number which specifies type of the message. All types are specified in the Table...

\subsection{Author}

Author field stores the id number of message's author. Each user has his unique id number.

\subsection{Group}

Group field says to which group on server message had been sent.

\subsection{Error code}

Error code is different from zero if message informs about error. By default it is set to zero which means everything is OK. 

\subsection{UTC time}

Seconds from epoch (1st January 1970, 0:00) in UTC zone. This value is always set by server.

\subsection{Size}

Message's data size.

\subsection{Options}

Optional field. Options are unused.

\subsection{Author's identification}

C-string in UTF-8 set by server.

\subsection{Client's identification

\end{document}
